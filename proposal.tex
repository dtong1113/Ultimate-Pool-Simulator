\documentclass[11pt]{article}
\usepackage[pdftex]{graphicx}
\usepackage{enumerate}
\usepackage[
    letterpaper,
    left=1.0in,
    right=1.0in,
    top=0.75in
]{geometry}

\setlength{\textwidth}{6.5in}
\setlength{\textheight}{10in}
\pagenumbering{gobble}
\begin{document}
\begingroup  
  \centering
  \LARGE \textbf{SE101 Lab Proposal: Digit Recognizer}\\\vspace{0.4cm}
  \large Daniel Tong dy2tong\\
  \large Jeffrey Xiao j39xiao\\\vspace{0.2cm}
  \normalsize \today \par
\endgroup

\vspace{0.5cm}
\section*{Summary}
Our idea is to use the accelerometer and gyroscope to track the position of the Tiva on a 3D plane. By frequently sampling the position of the Tiva, we can ``draw'' digits by mapping 3D coordinates to a 2D plane. We then plan to use machine learning (forward-feed neural networks) to classify the drawn digits from the 2D image. Finally we will output the digits in the visual display as well as in Morse code.

\section*{Software Components}
\begin{itemize}
	\item \textbf{Position Sampling}: Tracking and recording the 3D position of the Tiva over a period of a time using the accelerometer and gyroscope. The position is sampled at regular intervals when a specific button is held. The image is then fed to the data whitening step when another button is pressed.
	\item \textbf{Data Whitening}: Taking raw coordinates and processing it into a 28 pixel by 28 pixel image for image classification.
	\item \textbf{Neural Network}: Training a multilayer neural network with ten classes to classify digits.
	\item \textbf{Morse Code and Display}: Displaying the number on the LCD screen and flashing it in Morse code.
\end{itemize}

\section*{Hardware Components}
\begin{itemize}
	\item \textbf{Accelerometer}: For \emph{Position Sampling}
	\item \textbf{Gyroscope}: For \emph{Position Sampling}
	\item \textbf{LCD Screen}: For \emph{Morse Code and Display}
	\item \textbf{Buttons}: For \emph{Position Sampling}
\end{itemize}

\section*{Challenges}
\begin{itemize}
	\item Using the accelerometer and gyroscope to track the 3D position of the Tiva.
	\item Converting the 3D position to accurate relative 2D position.
	\item Training an efficient multi-layer neural network to recognize drawn digits.
	\item Using regularization during training to produce a more accurate model. Since our training data will be the MNIST dataset, it might not generalize well to the digits drawn by the Tiva.
	\item Properly displaying the number and Morse code to the LCD screen.
\end{itemize}

\end{document}